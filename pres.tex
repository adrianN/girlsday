\documentclass{beamer}
\usepackage{textcomp}
\usepackage[ngerman]{babel}
\usepackage[utf8]{inputenc}
\usepackage[T1]{fontenc}
\usepackage{fancybox}
\usepackage{graphicx}
\usepackage{hyperref}
\usepackage{amssymb}
\usepackage{amsmath}
\usepackage{amsthm}


\usetheme{Berlin}
\useoutertheme{infolines}
\usefonttheme{professionalfonts}
\usecolortheme{seahorse}
\usecolortheme{rose}

%\setbeamercovered{transparent}
\beamertemplatenavigationsymbolsempty
\setbeamertemplate{footline}[frame number]
\theoremstyle{example}
\newtheorem{ex}{Beispiel}
\renewenvironment{example}{\begin{ex}}{\end{ex}}


\title{Bilder In Weniger Als Tausend Worten}
\subtitle{Girlsday 2012}
\institute{Universit\"at des Saarlandes}
\date{}
\author[A. Neumann]{
	Adrian Neumann
}



\begin{document}
\frame{\titlepage}

\section{Rechnen}
\begin{frame}{Introduction}
Manipulation eines Speichers nach festen Regeln
\begin{itemize}
\item Mathe auf Papier
\item Programme in Computern
\end{itemize}
\begin{block}{}\centering
Informatiker untersuchen einfache Regelsysteme mit kompliziertem Verhalten
\end{block}
\end{frame}

\subsection{Beispiel:Game of Life}
\begin{frame}{Beispiel: Game of Life}
\begin{itemize}
\item Speicher: 
  \begin{itemize}
    \item 2D-Gitter
    \item Zellen sind ``lebendig'' (schwarz) oder ``tot'' (weiß)
  \end{itemize}
\item Regeln
  \begin{itemize}
  \item Lebendig $\rightarrow$ Tot: $<2$ oder $>3$ lebendige Nachbarn
  \item Lebendig $\rightarrow$ Lebendig: $2-3$ lebendige Nachbarn
  \item Tot $\rightarrow$ Lebendig: 3 lebendige Nachbarn
  \end{itemize}
\end{itemize}
\end{frame}

\begin{frame}{Beispiel: Game of Life}
  TODO schicke Bilder
\end{frame}

\section{Arbeiten mit ``Sprachen''}
\begin{frame}
  \begin{definition}
    \begin{description}
    \item[Alphabet] Endliche Menge von ``Buchstaben''
    \item[Wort] Endliche Sequenz von Buchstaben
    \item[Sprache] Menge von Wörtern (i.\,d.\,R.\ unendlich)
    \end{description}
  \end{definition}
  \begin{example}
    \begin{description}
    \item[Alphabet] $\Sigma = \{ (,)\}$
    \item[Wort] $()(())$ oder $((())$
    \item[Sprache] $L=\{w | \text{$w$ ist ein korrekt geklammert}\}$
    \end{description}
  \end{example}
\end{frame}

\subsection{Grammatiken}
\begin{frame}{Grammatiken}
  \begin{itemize}
  \item Einfache Regeln zum Beschreiben von Sprachen
  \item ``Grammatik'' wie in der Schule: Irreführend
  \item Zwei Arten von Symbolen:
    \begin{itemize}
    \item Buchstaben (\emph{Terminalzeichen}): hier Kleinbuchstaben $a,b,c$
    \item Platzhalter (\emph{Nichtterminalzeichen}): hier Großbuchstaben $A,B,C$

      Ein Nichtterminalzeichen ist besonders: Das \emph{Startsymbol}
    \end{itemize}
  \item Regeln um Wörter zu erzeugen
  \item Sprache: Menge aller Wörter, die man mit der Grammatik erzeugen kann
  \end{itemize}
\end{frame}

\subsection{Produktionsregeln}
\begin{frame}{Produktionsregeln}
  \begin{itemize}
  \item Regeln haben die Form
    \[\text{\textit{String}}_1 \longrightarrow \text{\textit{String}}_2\]
  \item Darf linke Seite durch rechte Seite ersetzen
  \item Angefangen wird mit dem Startsymbol
  \item Prozess ist abgeschlosesn, wenn kein Nichtterminal übrig ist
  \item Heute: \emph{Kontextfreie} Grammatiken

    \textit{String$_1$} ist genau ein Nichtterminal
  \end{itemize}
\end{frame}

\subsection{Beispiel: Klammerausdrücke}
\begin{frame}{Beispiel}
  \begin{itemize}
  \item Schon gesehen:$\Sigma = \{(,)\}$ und  $L = \{ w | \text{$w$ ist korrekt geklammert}\}$
  \item Grammatik:
    \begin{columns}[l]
     \column{2cm}
    \begin{align*}
      S &\rightarrow ()\\
      S &\rightarrow SS\\
      S &\rightarrow (S)
    \end{align*}
     \column{6cm}
     \begin{ex}
       \vspace*{5cm}
       \hspace*{6cm}
     \end{ex}
    \end{columns}
  \end{itemize}
\end{frame}

\subsection{Mehr als Buchstaben}
\begin{frame}{Mehr als Buchstaben}
  \begin{itemize}
  \item Besonderes Alphabet
    \begin{itemize}
    \item Geometrische Primitive (Rechteck, Dreieck, Ellipse)
    \item Rotationen
    \item Skalierungen
    \item Translationen
    \item Farbveränderungen
    \end{itemize}
  \item Grammatik regelt, wie daraus Bilder erzeugt werden
  \end{itemize}
\end{frame}
\end{document}